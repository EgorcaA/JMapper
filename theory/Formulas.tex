\documentclass[12pt,a4paper]{article}
% add (renew) commands

\usepackage[margin=1.8cm]{geometry}         %!  фиксирует оступ на 2cm
\usepackage{xcolor}
\usepackage{hyperref} 
\definecolor{grey}{HTML}{666666}
\definecolor{linkcolor}{HTML}{0000CC}
\definecolor{urlcolor}{HTML}{006600}
\hypersetup{
    pdfstartview=FitH,  
    linkcolor=linkcolor,
    urlcolor=urlcolor, 
    colorlinks=true,
    citecolor=blue,
    unicode=true}
\usepackage[utf8]{inputenc}
\usepackage[english]{babel}
\usepackage{graphicx}%для пикч
\usepackage{pgfplots}%для графиков
\pgfplotsset{compat=1.9}%тоже
\usepackage{amsfonts} %ШРИФТ ДЛЯ R
\usepackage{float} %размещение пикч
\usepackage{amsmath} %для aligned тип чтобы равно было посередине
\usepackage{wrapfig}
\usepackage{mathtext}
\usepackage{gensymb}
\usepackage{enumitem}
\usepackage{tabto}
\usepackage{tcolorbox}
\usepackage{subcaption}

\newcommand*\diff{\mathop{}\!\mathrm{d}}
\newcommand*\bigcdot{\mathpalette\bigcdot@{.5}}

\usepackage{listings}
\usepackage{color}

\definecolor{dkgreen}{rgb}{0,0.6,0}
\definecolor{gray}{rgb}{0.5,0.5,0.5}
\definecolor{mauve}{rgb}{0.58,0,0.82}

\lstset{frame=tb,
extendedchars=\true,
  language=sql,
  aboveskip=3mm,
  belowskip=3mm,
  showstringspaces=false,
  columns=flexible,
  basicstyle={\small\ttfamily},
  numbers=none,
  numberstyle=\tiny\color{gray},
  keywordstyle=\color{blue},
  commentstyle=\color{dkgreen},
  stringstyle=\color{mauve},
  breaklines=true,
  breakatwhitespace=true,
  tabsize=3
}
\graphicspath{{./pics/}}
\begin{document}


\begin{center}
    \LARGE \textsc{formulas}
\end{center}

\hrule

\phantom{42}

\begin{flushright}
    \begin{tabular}{rr}
    % written by:
        \textbf{Authors}: 
        & EA AK\\
    % date:
        \textbf{From}: &
        \textit{\today}\\
    \end{tabular}
\end{flushright}

\thispagestyle{empty}


\section{Theory}
The exchange interactions in DFT were estimated by magnetic force theorem (MFT)  approach \cite{korotin_calculation_2015,liechtenstein_local_1987} using electron Green's functions, 
calculated in Wannier basis 
%consisting of d states of Cr and p states of Te 
\begin{equation}
    G_{rr', \sigma}^{m m^{\prime}}(\mathbf{k},i \nu_n)=\left[(i \nu_n + \mu)I-H^{rm, r'm^{\prime}}_{\mathbf{k},\sigma}\right]_{rm,r'm'}^{-1}
\end{equation} 
where $I$ is the identity matrix,
 $H^{rm, r'm^{\prime}}_{\mathbf{k},\sigma}$ is the Wannier Hamiltonian, 
\[
    H^{rm, r'm^{\prime}}_{\mathbf{k},\sigma} =\left\langle\psi_{\mathbf{k} \sigma rm}^{\mathrm{W}}|H| \psi_{\mathbf{k} \sigma r'm^{\prime}}^{\mathrm{W}}\right\rangle=
    \sum_{\mathbf{R}} e^{i \mathbf{k} \cdot \mathbf{R}}\langle\mathbf{0} \sigma rm |H| \mathbf{R} \sigma r'm^{\prime}\rangle
\]
and the inversion is performed in the site- and orbital space. 
To obtain $H^{rm, r'm^{\prime}}_{\mathbf{k},\sigma}$ wannier transformation was performed on both spin components separately 
relying on that $| \mathbf{R} \sigma rm\rangle $ would be independent on spin $| \mathbf{R} rm\rangle$ so that we would have common basis for both components. 
%where $ r, r'$ numerates atoms in the unit cell, 

\begin{equation}
    \label{JqDFT}
    J^{rr'}_{\mathbf{q}}=  -\frac{2T}{{\mathfrak m}_r{\mathfrak m}_{r'}}  \sum_{{\mathbf k},i \nu_n}\operatorname{Tr}
    \left[\Delta_{ r} \tilde{G}_{ rr',\downarrow}({\mathbf k}+ {\mathbf q},i \nu_n)\right. \times \left.\Delta_{r'} \tilde{G}_{r'r,\uparrow}({\mathbf k},i \nu_n) \right]
\end{equation}

where $\Delta_{r}^{m m^{\prime}} = \sum_{\mathbf k}(H_{\mathbf{k},\uparrow }^{rm, rm^{\prime}}-H_{\mathbf{k},\downarrow}^{rm, rm^{\prime}})$ is spin splitting 
and $\tilde{G}^{mm'}_{ rr',\sigma}({\bf k},i \nu_n)={G}^{mm'}_{ rr',\sigma}({\bf k},i \nu_n)-
%{G}^{mm'}_{{\rm loc},r,\sigma}(i \nu_n)$,
%J_{\text{self}}$ is self exchange that can be found as $J_{\text{self}} = -\frac{T}{4}  \sum_{i \nu_n} \operatorname{Tr} \left[\Delta_{ i i} G_{\text{loc}, \downarrow}(i \nu_n) \Delta_{i i} G_{\text{loc}, \uparrow}(i \nu_n) \right]$ 
%$G_{\text{loc},r,\sigma}^{mm'} (i\nu_n)= 
\sum_{{\mathbf k}'} G_{rr, \sigma }^{m m^{\prime}}(\mathbf{k}',i \nu_n)$, the trace is taken over orbital indexes, and $\mathfrak{m}_r$ is the magnetic moment at the $r$-th atom (in units of Bohr magneton $\mu_B$). 
Equivalently we calculate 
\begin{equation}
    J^{rr'}_{\mathbf{q}}=  -\frac{2T}{{\mathfrak m}_r{\mathfrak m}_{r'}}  \sum_{{\mathbf k},i \nu_n}\operatorname{Tr}
    \left[\Delta_{ r} G_{ rr',\downarrow}({\mathbf k}+ {\mathbf q},i \nu_n)\right. \times \left.\Delta_{r'} G_{r'r,\uparrow}({\mathbf k},i \nu_n) \right] - J_{\text{self}} \delta_{rr'}
\end{equation}
where $J_{\text{self}}$ is self exchange that can be found as $J_{\text{self}} = -\frac{2T}{{\mathfrak m}_r{\mathfrak m}_{r'}}  \sum_{{\mathbf k},i \nu_n}\operatorname{Tr}
\left[\Delta_{ r} G_{ \text{loc},\downarrow}(i \nu_n)\right. \times \left.\Delta_{r'} G_{\text{loc},\uparrow}(i \nu_n) \right]$ 
with $G_{\text{loc}, \sigma} = \frac{1}{V_{BZ}} \int_{B Z} G_{i i, \sigma }^{m m^{\prime}}(i \nu_n, \mathbf{k})  d \mathbf{k}$.



% Nearest neighbours interaction can be found similarly as 
% \begin{align}
%    J_{i j}(\mathbf{R}_{l})  = -\frac{T}{4}  \sum_{i \nu_n}\notag\\ \times\operatorname{Tr}
%    \left[\Delta_{ i i} G_{i j, \downarrow}(i \nu_n, \mathbf{R}_{l}) \Delta_{j j} G_{j i, \uparrow}(i \nu_n, -\mathbf{R}_{l}) \right]
% \end{align}
% where $G_{i j, \sigma}(i \nu_n, \mathbf{R}_{l})$ is Fourier image from $G_{i j, \sigma}^{m m^{\prime}}(i \nu_n, \mathbf{k})$. 


\subsection{1 atom per cell} % (fold)
\label{sub:1}
In case of one magnetic atom in unit cell we calculate 
$    G_{0 j, \sigma}^{m m^{\prime}}(i \nu_n)= \frac{1}{V_{BZ}} \int_{B Z} G_{\sigma}^{m m^{\prime}}(i \nu_n, \mathbf{k}) e^{2 \pi i \mathbf{k} \mathbf{R}_{j}} d \mathbf{k}$ 
and $G_{ j 0, \sigma}^{m m^{\prime}}(i \nu_n)= \frac{1}{V_{BZ}} \int_{B Z} G_{\sigma}^{m m^{\prime}}(i \nu_n, \mathbf{k}) e^{-2 \pi i \mathbf{k} \mathbf{R}_{j}} d \mathbf{k}$
where $k$ is in range $[0, 1] \otimes [0, 1]$ and $\mathbf{R}_j$ is integer vector, $\Delta^{m m^{\prime}} = \frac{1}{V_{BZ}} \int_{B Z}\left(H_{ \uparrow}^{m m^{\prime}}(\mathbf{k})-H_{ \downarrow}^{m m^{\prime}}(\mathbf{k})\right) d \mathbf{k}$

% \[
%     J_{0 j}= -\frac{2T}{{\mathfrak m} {\mathfrak m}}  \sum_{i \nu_n} \sum_{\substack{m m^{\prime} \\ m^{\prime \prime} m^{\prime \prime \prime}}} 
%     \left(\Delta^{m m^{\prime}}  G_{0 j, \downarrow}^{m^{\prime} m^{\prime \prime}} (i \nu_n)  \Delta^{m^{\prime \prime} m^{\prime \prime \prime}}   G_{j 0, \uparrow}^{m^{\prime \prime \prime} m} (i \nu_n) \right) =
%     -\frac{2T}{{\mathfrak m} {\mathfrak m}}   \sum_{i \nu_n} \operatorname{Tr}
%     \left[\Delta G_{0 j, \downarrow}(i \nu_n) \Delta G_{j 0, \uparrow}(i \nu_n) \right]
% \]
\begin{multline*}
    J_{0 j}= -\frac{2T}{{\mathfrak m} {\mathfrak m}}  \sum_{i \nu_n} \sum_{\substack{m m^{\prime} \\ m^{\prime \prime} m^{\prime \prime \prime}}} 
    \left(\Delta^{m m^{\prime}}  G_{0 j, \downarrow}^{m^{\prime} m^{\prime \prime}} (i \nu_n)  \Delta^{m^{\prime \prime} m^{\prime \prime \prime}}   G_{j 0, \uparrow}^{m^{\prime \prime \prime} m} (i \nu_n) \right) =\\
    -\frac{2T}{{\mathfrak m} {\mathfrak m}}   \sum_{i \nu_n} \operatorname{Tr}
    \left[\Delta G_{0 j, \downarrow}(i \nu_n) \Delta G_{j 0, \uparrow}(i \nu_n) \right]
\end{multline*}

\begin{multline*}
    J_{q}=-\frac{2T}{{\mathfrak m}^2  V_{BZ}}  \sum_{i \nu_n} \int_{1 BZ} \diff k \sum_{\substack{m m^{\prime} \\ m^{\prime \prime} m^{\prime \prime \prime}}}  
    \left(\Delta^{m m^{\prime}}  G_{k + q, \downarrow}^{m^{\prime} m^{\prime \prime}} (i \nu_n)  \Delta^{m^{\prime \prime} m^{\prime \prime \prime}}   G_{k, \uparrow}^{m^{\prime \prime \prime} m} (i \nu_n) \right) = \\
    -\frac{2T}{{\mathfrak m}^2  V_{BZ}}  \sum_{i \nu_n}  \int_{1 BZ} \diff k \operatorname{Tr}
    \left[\Delta G_{k+ q, \downarrow}(i \nu_n) \Delta G_{k, \uparrow}(i \nu_n) \right]
\end{multline*}


\begin{tcolorbox}
\textbf{Tips to check}
    \begin{multline*}
        G_{0 j, \sigma}^{m m^{\prime}}( - i \omega_n)^{\dagger} = \frac{1}{V_{BZ}} \int_{B Z} G_{\sigma}^{m m^{\prime}}(- i \omega_n, \mathbf{k})^{\dagger} e^{- 2 \pi i \mathbf{k} \mathbf{R}_{j}} d \mathbf{k} = \\
        \frac{1}{V_{BZ}} \int_{B Z} G_{\sigma}^{m m^{\prime}}( i \omega_n, \mathbf{k}) e^{- 2 \pi i \mathbf{k} \mathbf{R}_{j}} d \mathbf{k} = G_{j 0, \sigma}^{m m^{\prime}}(  i \omega_n)
    \end{multline*}

    \[
    0 = \int \diff q J_{q}
    \]

\end{tcolorbox}

Magnon dispersion in Heisenberg model
\[
    E(\mathbf{q})=\frac{4 \mu_B}{{\mathfrak m}}[J(\mathbf{0})-J(\mathbf{q})]
\]



\subsection{2 atoms per cell} % (fold)
\label{sub:1}
In case of 2 (magnetic) atoms in unit cell we add indices $ i, j$ that number these atoms
\[
    G_{i j, \sigma}^{m m^{\prime}}(i \nu_n, \mathbf{k})=\left(i \nu_n + E_F-H_{m m^{\prime}, i j, \sigma}^{W F}(\mathbf{k})\right)^{-1}
\]
\begin{align*}
    G_{0 0, \sigma}^{m m^{\prime}}(i \nu_n, \mathbf{R}_{l}) &= \frac{1}{V_{BZ}} \int_{B Z} G_{\sigma 0 0}^{m m^{\prime}}(i \nu_n, \mathbf{k}) e^{2 \pi i \mathbf{k} \mathbf{R}_{l}} d \mathbf{k}\\
    G_{0 1, \sigma}^{m m^{\prime}}(i \nu_n, \mathbf{R}_{l}) &= \frac{1}{V_{BZ}} \int_{B Z} G_{\sigma 0 1}^{m m^{\prime}}(i \nu_n, \mathbf{k}) e^{2 \pi i \mathbf{k} \mathbf{R}_{l}} d \mathbf{k} 
\end{align*}
where $\mathbf{R}_{l}$ is translation vector (in our case both atoms are in G so $\mathbf{R}_{l}$ is integer vector)
\begin{equation}
    \Delta_{i i}^{m m^{\prime}} = \frac{1}{V_{BZ}} \int_{B Z}\left(H_{ \uparrow i i}^{m m^{\prime}}(\mathbf{k})-H_{ \downarrow i i}^{m m^{\prime}}(\mathbf{k})\right) d \mathbf{k} .
\end{equation}

\begin{equation}
    J_{i j}(\mathbf{R}_{l})  = \frac{T}{4}  \sum_{i \nu_n} \operatorname{Tr}
    \left[\Delta_{ i i} G_{i j, \downarrow}(i \nu_n, \mathbf{R}_{l}) \Delta_{j j} G_{j i, \uparrow}(i \nu_n, -\mathbf{R}_{l}) \right]
\end{equation}
\begin{equation}
    J_{i j} (q) =  \frac{T}{4 V_{BZ}}  \sum_{i \nu_n}  \int_{1 BZ} \diff k \operatorname{Tr}
    \left[\Delta_{ i i} G_{ i j, \downarrow}(i \nu_n, k+ q) \Delta_{j j} G_{j i, \uparrow}(i \nu_n, k) \right]
\end{equation}
from $J_{i j} (\mathbf{q})$ we can also get $J_{i j}(\mathbf{R}_{l})$ via 
\begin{equation}
    J_{i j}(\mathbf{R}_{l}) = \frac{1}{V_{BZ}} \int_{B Z} J_{i j} (\mathbf{q}) e^{2 \pi i \mathbf{q} \mathbf{R}_{l}} d \mathbf{q}
\end{equation}




% \nocite{*}

\bibliographystyle{ieeetr}
\bibliography{bibl.bib}



\end{document}